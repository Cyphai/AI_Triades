\chapter{Informations générales}
\section{Première utilisation et mise à jour du datapack}

Lors du premier lancement de la version étudiante de \tria, l'utilisateur est invité à saisir une adresse à laquelle le datapack peut être téléchargé. En effet, le logiciel en version étudiante est initialement fourni sans datapack afin de permettre un déploiement plus aisé de nouvelles sessions ou d'une traduction locale du contenu du datapack. Il est possible qu'il soit nécessaire de relancer le logiciel une fois le fichier téléchargé.\\

\subsubsection{Programme AutoUpdater}
Le programme "AutoUpdater" permet de mettre à jour automatiquement le datapack à partir d'un fichier téléchargé sur internet.\\

Lors du lancement de l'AutoUpdater, l'utilisateur est invité à saisir l'adresse de téléchargement du datapack. Dans le cas où le datapack courant et le nouveau contiennent une traduction ou des sessions portant le même nom, le programme demande à l'utilisateur quelle version il souhaite conserver.\\

Une sauvegarde de l'ancien datapack est automatiquement réalisée et placée dans le dossier "datapack\_backup".


\subsection{Gestion de la licence d'utilisation}
Les datapacks fournis avec la version étudiante du logiciel ont un durée d'utilisation limitée dans le temps. Lors de l'achat d'une licence, un code de déblocage du datapack est fourni. Ce code peut être saisi à l'aide de l'option "Enregistrer une licence". Une fenêtre s'ouvre alors invitant l'utilisateur à saisir l'adresse mail associée à la licence puis le code de validation.\\

Lors de l’utilisation d'un datapack dont la durée d'essai s'est achevé, une boite de dialogue invite l'utilisateur à saisir ces mêmes informations pour permettre l'utilisation du datapack. Il est possible qu'un redémarrage du logiciel soit nécessaire lors de la saisie d'une licence.\\
