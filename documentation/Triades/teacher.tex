
\chapter{Fonctions dédiées à la version enseignant}
Le menu "Outils", uniquement présent dans la version enseignant du logiciel, permet de générer des licences d'utilisation ainsi que d'exporter des datapack à destination des versions étudiantes.

Lors de l'utilisation de ces deux outils, il vous est demandé de saisir votre clé personnelle. Cette dernière vous a normalement été fournie lors de la livraison du logiciel.\\

\section{Génération de licences}
La génération de licences se déroule en deux étapes. Tout d'abord, l'utilisateur doit saisir la liste des adresses mails pour lesquelles une licence doit être générée. Il est possible d'extraire cette liste depuis le presse-papier. Dans ce cas, les séparateurs utilisés sont au choix : la virgule, le point-virgule, la tabulation et le retour chariot. Cela permet en particulier d'importer facilement cette liste depuis un tableur.\\
Il est aussi possible d'ajouter une adresse à l'aide du champ mail et du bouton juste à sa droite situé en haut de la fenêtre. Les différentes entrées pouvant être modifiées et supprimées à l'aide des boutons dédiés en haut à droite de la fenêtre.\\
Le type de licence peut être choisi à l'aide de la liste à puce en bas de la fenêtre. Il n'est pas possible de spécifier plusieurs licences différentes pour une même liste. L'option "Licence déjà facturée" permet de générer à nouveau un code de licence au cas ou ce dernier aurait été perdu par l'utilisateur.\\

L'étape suivante consiste à générer les licences. Attention, un connexion internet sera nécessaire lors de cette étape. Le processus est lancé lors du clic sur le bouton "Générer les licences". La liste de mail ne sera plus modifiable par la suite.\\

L'utilisateur est invité à choisir le lieu d'enregistrement du fichier de licence dans lequel il veut stocker le récapitulatif des licences générés. Il est ensuite possible de copier au choix une adresse mail, un code de déblocage, un couple mail/code de déblocage ou la liste entière à l'aide des boutons situés sur la droite de la fenêtre.\\

Une fois la fenêtre de génération des licences fermées, il n'est plus possible d'accéder à la liste générées au sein du logiciel, mais uniquement grâce au fichier récapitulatif créé lors de la génération.\\

Une demande de facturation est automatiquement envoyé à l'équipe de développement lors de la génération des licences.

\section{Génération d'un datapack pour les version étudiantes}

L'autre option du menu outils permet de générer des datapack pour les versions étudiantes du logiciel. Lors de cette opération, l'utilisateur est tout d'abord invité à saisir la durée de validité du datapack. La durée maximale est de 365 jours. La durée minimale est 0 jours, cela correspond à un datapack bloqué immédiatement. Ce type de datapack est destiné à être mis à disposition des utilisateurs ayant acheté une licence d'utilisation afin qu'(il puisse le télécharger lors d'une réinstallation ultérieure du logiciel.\\

La fenêtre suivante invite l'utilisateur à choisir l'endroit ou le datapack devra être enregistré.\\

Une fois l'emplacement choisi, le datapack est automatiquement exporté.\\


