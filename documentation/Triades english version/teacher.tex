
\chapter{Teachers version features}

The "Tools" menu, only available in the teacher version, allow user to generate use-licenses and limited time datapack for student version. 
%Le menu "Outils", uniquement présent dans la version enseignant du logiciel, permet de générer des licences d'utilisation ainsi que d'exporter des datapack à destination des versions étudiantes.

Those two tools request user to enter his personnal key. This key has been normally given to you during software distribution. If not, please contact developpers team.\\
%Lors de l'utilisation de ces deux outils, il vous est demandé de saisir votre clé personnelle. Cette dernière vous a normalement été fournie lors de la livraison du logiciel.\\

\section{Licenses generation}

The first step of licenses generation is to enter the list of mail address for which a license needs to be generated. This list could be extract from the clipboard, the mail address should be separated by ',' or ';' or a tabulation or a line jump. This function allow to import the list from a spreadsheet software.\\
%La génération de licences se déroule en deux étapes. Tout d'abord, l'utilisateur doit saisir la liste des adresses mails pour lesquelles une licence doit être générée. Il est possible d'extraire cette liste depuis le presse-papier. Dans ce cas, les séparateurs utilisés sont au choix : la virgule, le point-virgule, la tabulation et le retour chariot. Cela permet en particulier d'importer facilement cette liste depuis un tableur.\\

It is also possible to add an address with the dedicated field and button in the top of the window. Entries can be modified or deleted with the button on the top right of the window.\\
%Il est aussi possible d'ajouter une adresse à l'aide du champ mail et du bouton juste à sa droite situé en haut de la fenêtre. Les différentes entrées pouvant être modifiées et supprimées à l'aide des boutons dédiés en haut à droite de la fenêtre.\\

License type could be selected with the radio-button list on the bottom of the window. It is not possible to specify different licence for a same list. The option "Already billed licenses" allow to regenerate an unlock key for an user who have lost his key.\\
%Le type de licence peut être choisi à l'aide de la liste à puce en bas de la fenêtre. Il n'est pas possible de spécifier plusieurs licences différentes pour une même liste. L'option "Licence déjà facturée" permet de générer à nouveau un code de licence au cas ou ce dernier aurait été perdu par l'utilisateur.\\

The next step is to generate licences. Warning, an internet connection is requested during this step. The process is launched with the button "Generate licences" in the bottom right of the window. The mail list won't be editable further.\\
%L'étape suivante consiste à générer les licences. Attention, un connexion internet sera nécessaire lors de cette étape. Le processus est lancé lors du clic sur le bouton "Générer les licences". La liste de mail ne sera plus modifiable par la suite.\\

The user have to select the file which a summary of all generated licenses have to be saved. Then , it is possible to copy a mail address, an unlock key, a couple mail/key or the while list with the buttons on the right of the window.\\
%L'utilisateur est invité à choisir le lieu d'enregistrement du fichier de licence dans lequel il veut stocker le récapitulatif des licences générés. Il est ensuite possible de copier au choix une adresse mail, un code de déblocage, un couple mail/code de déblocage ou la liste entière à l'aide des boutons situés sur la droite de la fenêtre.\\

Once the window has been closed, it is not possible to access to the generated key list in the software, but only by the summary file saved during the process.\\
%Une fois la fenêtre de génération des licences fermées, il n'est plus possible d'accéder à la liste générées au sein du logiciel, mais uniquement grâce au fichier récapitulatif créé lors de la génération.\\


A billing request will be automatically send to the software dealer during licenses generations.\\
%Une demande de facturation est automatiquement envoyé à l'équipe de développement lors de la génération des licences.


\section{Generation of limited time datapack for student version}%Génération d'un datapack pour les version étudiantes}

The other option of tools menu allow to generate trial datapack. In this purpose, the user has to enter the trial period duration. The maximum is 365 days, and the minimum is 0. The trial period start on the datapack generation day. If the value 0 is used, the datapack will be always locked. It is usefull for user who had bought a license and who needs to reinstall the software. They can download a datapack of this kind and then unlock it whit their license key.\\
%L'autre option du menu outils permet de générer des datapack pour les versions étudiantes du logiciel. Lors de cette opération, l'utilisateur est tout d'abord invité à saisir la durée de validité du datapack. La durée maximale est de 365 jours. La durée minimale est 0 jours, cela correspond à un datapack bloqué immédiatement. Ce type de datapack est destiné à être mis à disposition des utilisateurs ayant acheté une licence d'utilisation afin qu'(il puisse le télécharger lors d'une réinstallation ultérieure du logiciel.\\

After setting the trial duration, the user have to select where to save the trial datapack. Then, the datapack will be automatically exported.\\
%La fenêtre suivante invite l'utilisateur à choisir l'endroit ou le datapack devra être enregistré.\\

%Une fois l'emplacement choisi, le datapack est automatiquement exporté.\\

